\documentclass[12pt,a4paper,oneside]{article}
\usepackage[T1]{fontenc}
\usepackage[utf8]{inputenc}
\usepackage{amsmath,amssymb}
\usepackage[russian,english]{babel}
\usepackage{tikz}
\usepackage{verbatim}
\begin{document}
\selectlanguage{russian}
\begin{center}
САНКТ-ПЕТЕРБУРГСКИЙ ГОСУДАРСТВЕННЫЙ\newline
ПОЛИТЕХНИЧЕСКИЙ УНИВЕРСИТЕТ\newline
Физика, 2014 Г.\newline
Вариант ???\newline
\end{center} 
1. Среди существующих 220 видов черепах самой быстрой является кожистая черепаха. Её максимальная скорость передвижения в воде достигает 35 км/ч. Эта величина в м/с равна ...\newline
A) \(\approx\)9,7. Б) 3,5. В) \(\approx\)97. Г) 35. \newline
\begin{minipage}{0.72\textwidth}
2. На рисунке схематически изображена лестница AB, опирающаяся о стену. Величина момента силы тяжести \(m\vec{g}\), действующего на лестницу, относительно точки B равна ...\newline
\end{minipage}
\begin{minipage}{0.24\textwidth}
\begin{flushright}
\begin{tikzpicture}[scale=3]
  \coordinate [label={below:$B$}] (B) at (0, 0);
  \coordinate [label={above:$C$}] (C) at (0.5, 0.5);
  \coordinate [label={right:$A$}] (A) at (1, 1);
  \coordinate [label={below:$D$}] (D) at (0.5, 0);
  \coordinate [label={left:$m\vec{g}$}](M) at (0.5, 0.25);
  \coordinate (X) at (1, 0);
  \coordinate (Y) at (-0.5, 0);
  
  \draw [very thick] (A) -- (C) -- (B);
  \draw [->, very thick] (C) -- (M);
  \draw [dashed, very thick] (M) -- (D);
  \draw [dashed, very thick] (A) -- (X) -- (Y);


\end{tikzpicture}
\end{flushright}
\end{minipage}\newline
А) \(mg \cdot CB\) Б) \(mg \cdot DB\) В) \(mg \cdot CD\) Г)\(mg \cdot AB\)\newline
3. Температура плавления серебра \(960^{\circ}\)C. В градусах шкалы Кельвина это ...\newline
А) 687 К. Б) 960 К. В) 1060 К. Г) 1233 К.\newline
4. От капли воды, обладающей электрическим зарядом -2e, отделилась маленькая капля с зарядом +3e. Каким стал электрический заряд оставшейся части капли?\newline
А)+e. Б)-e. В) -5e Г) +5e\newline
5. С какой силой действует однородное магнитное поле индукцией 2,5 Тл на проводник длинной 50см, расположенный под углом \(30^{\circ}\) к вектору индукции, при силе тока в проводнике 0,5 А?\newline
6. Тело, двигаясь прямолинейно с ускорением 2 м/с\(^2\), за 0,1 минуту прошло путь 42 м. Какой была начальная скорость?\newline
7. Под действием какой горизонтальной силы вагонетка массой 350 кг движется по горизонтальным рельсам с ускорением 0,15 м/с\(^2\)\newline
8. В процессе эксперимента внутрення энергия газа уменьшилась на 40 кДж, и он совершил работу 35 кДж. Какое количество теплоты в результате теплообмена газ отдал окружающей среде?\newline
9. На каком расстоянии от заряда \(8 \cdot 10^{-6}\) Кл напряжённость электрического поля равна \(8 \cdot 10^5\) В/м. \newline
10. Фокусное расстояние линзы 20см. Расстояние от предмета до линзы равно 10см. Определить расстояние от изображения до линзы, если линза а) собирающая, б) рассеивающая. Какое получится изображение?\newline



\end{document}